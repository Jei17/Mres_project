\documentclass{beamer}
\usetheme{Singapore}
\title{Investigating the robustness of detecting truncation in rainfall data}
%\subtitle{A story about extremes}
\author{Jennifer Israelsson}
\institute{University of Reading}
\date{\today}

\begin{document}
	\begin{frame}
	\titlepage
	\end{frame}

\begin{frame}
\frametitle{Outline}
\tableofcontents
\end{frame}

\section{Background}

\begin{frame}
\frametitle{Introduction}
\begin{itemize}
	\item Truncation exists in many places
	\item Insurance for agriculture in developing countries
	\item Estimate the maximum daily rainfall
\end{itemize}
\end{frame}

\section{Extreme value theory}

\begin{frame}
\frametitle{Traditional vs extreme value statistics}
\begin{itemize}
	\item Traditional statistics
		\begin{itemize}
			\item Use all data
			\item Most weight to the bulk of the data
			\item Difficult to extrapolate
		\end{itemize}
	\item Extreme value statistics
		\begin{itemize}
			\item Only uses the most extreme data points
			\item Not influenced by the bulk of the data 
			\item Can extrapolate
		\end{itemize}
\end{itemize}
\end{frame}

\begin{frame}
\frametitle{Extreme value}
\begin{itemize}
	\item Block maxima method
	\begin{itemize}
		\item Use only the largest value in each block, e.g month or year
		\item Wastes data
	\end{itemize}
	\item Peak over threshold (POT)
		\begin{itemize}
			\item Uses all data points over a given high threshold
			\item Bias vs variance
		\end{itemize}
\end{itemize}
\end{frame}

\begin{frame}
\frametitle{Peak over threshold, POT}
\begin{itemize}
	\item Given that we have found a suitable threshold \textit{u},
	\begin{equation*}
	F_u(x) = \mathbb{P}(X - u \leq x | X > u) = \frac{F(x + u) - F(u)}{1 - F(u)}
	\label{excessdist}
	\end{equation*}
	\item Mean excess function,
	\begin{equation*}
	e(u) = \mathbb{E}(X - u | X > u)
	\label{mefunc}
	\end{equation*}
\end{itemize}
\end{frame}

\begin{frame}
\frametitle{General Pareto distribution, GPD}
\begin{itemize}
	\item The distribution $F_u(x)$ converges to the \textit{General pareto distribution, GPD} as $ n \rightarrow \infty$
	\begin{equation*}
	G_{\xi, \nu, \sigma}(x) =
	\begin{cases}
	1-(1+\xi \frac{x-\nu}{\sigma})^{-1/\xi}, & \text{if} \ 1+\xi \frac{x-\nu}{\sigma} > 0, \quad \xi \neq 0 \\
	1- e^{-\frac{x-\nu}{\sigma}}, & \text{if} \  \xi = 0
	\end{cases}
	\label{GPD}
	\end{equation*}
	\item $\xi$ is called the shape parameter
\end{itemize}
\end{frame}

\begin{frame}
\frametitle{Choose threshold}
\begin{itemize}
	\item Mean excess plot
	\begin{equation*}
	e(u) = \frac{\beta + \xi u}{1 - \xi}, \quad  u \in D(\xi,\beta),  \quad \xi < 1.
	\end{equation*}
	with,
	\begin{equation*}
	D(\xi, \beta) =
	\begin{cases}
	[0,\infty], & \text{if} \ \xi \geq 0 \\
	[0, \frac{\beta}{\xi}], & \text{if} \ \xi < 0 \\
	\end{cases}
	\end{equation*}
	\item Shape plot
		\begin{figure}
		\includegraphics[width=0.4\linewidth]{ShapeMid.pdf}
	\end{figure}
\end{itemize}
\end{frame}

\begin{frame}
\frametitle{Truncated distribution}
\begin{itemize}
	\item The unrestricted can lead to unphysical estimations
	\item If truncated at $T > 0$, and $X =_d Y | Y < T$
	\item Truncated POT
	\begin{equation*}
	\begin{split}
	\mathbb{P}\left(\frac{X - t}{\sigma _t} > x | X > t\right) & = \mathbb{P}\left(\frac{Y - t}{\sigma _t} > x | t < Y < T\right) \\
	& = \frac{\frac{\mathbb{P}(Y > t+x\sigma _t )}{\mathbb{P}(Y>t)} -  \frac{\mathbb{P}(Y > T)}{\mathbb{P}(T > t)}}{1 - \frac{\mathbb{P}(Y > T)}{\mathbb{P}(T>t)}}
	\end{split}
	\label{tPOT}
	\end{equation*}
\end{itemize}
\end{frame}

\begin{frame}
\frametitle{Goodness-of-fit test}
\begin{itemize}
	\item Different truncations
	\begin{itemize}
		\item Rough truncation with the threshold $t = t_n$
			\begin{equation*}
			\frac{T - t}{\sigma_t} \rightarrow \kappa > 0
			\label{RoughTru}
			\end{equation*}
			\begin{equation*}
			\frac{\mathbb{P}(Y > T)}{\mathbb{P}(Y > t)} \rightarrow (1+ \xi \kappa)^{-1/\xi}
			\end{equation*}
		\item Light truncation with the threshold $t= t_n$ :
		\begin{equation*}
		\frac{\mathbb{P} (Y > T)}{\mathbb{P}(Y>t)} \rightarrow 0
		\label{LightTru}
		\end{equation*}	
	\end{itemize}
	\item Goodness-of-fit test
	\begin{equation*}
	T_{k,n} := k(1 + \hat{\tau}_k E_{1,k})^{-1/\hat{\xi}_k} > \text{log}(1/q)
	\label{gofTest}
	\end{equation*}
\end{itemize}
\end{frame}

\section{Weather generator}

\begin{frame}
\frametitle{Rainfall data}
\begin{columns}
	\column{0.5\linewidth}
	\begin{itemize}
		\item Used measured rainfall data to tune the weather generator
		\item One station and season used 
		\item Stationarity analysis
	\end{itemize}
	\column{0.5\linewidth}
	\begin{figure}
		\includegraphics[width=\linewidth]{images/mapRainmode.pdf}
		\caption{Rainmode distribution}
	\end{figure}
\end{columns}
\end{frame}

\begin{frame}
\frametitle{Construction}
\begin{itemize}
	\item One-step Markov chain for occurence
	\begin{itemize}
		\item e.g  probability of rain after rainy day 
	\end{itemize}
	\item Gamma distribution for bulk of data
	\begin{equation*}
	f(x ; \alpha, \theta) = \frac{x^{\alpha-1} e^{-\frac{x}{\theta}}}{\theta^\alpha \Gamma(\alpha)}, \quad x > 0
	\end{equation*}
	\item GPD for extremes
	\item Mixed distribution 
	\begin{equation*}
	F(x)= F(u) \mathbb{P}(X \leq x | X \leq u) + (1-{F(u)}) \mathbb{P}(X \leq x | X > u)
	\end{equation*}
\end{itemize}	
\end{frame}

\section{Results}

\begin{frame}
\frametitle{Simulation study}
Using the 100 largest values and estimating the endpoint with the 0.9999 quantile of the GPD and a  positive $\xi$.
\begin{figure}
	\includegraphics[width=0.7\linewidth]{images/SimADAJJAk100.pdf}
	\caption{Base setup}
\end{figure}
\end{frame}

\begin{frame}
\frametitle{Simulation study}
Using the 0.999 quantile and the 0.99999 quantile as the endpoint
\begin{columns}
	\column{0.5\linewidth}
	\begin{figure}
		\includegraphics[width=\linewidth]{images/SimADAJJA999.pdf}
		\caption{0.999}
	\end{figure}
	\column{0.5\linewidth}
	\begin{figure}
		\includegraphics[width=\linewidth]{images/SimADAJJA99999.pdf}
		\caption{0.99999}
	\end{figure}
\end{columns}
\end{frame}

\begin{frame}
\frametitle{Simulation study}
Changing the parameterisation of the gamma distribution and GPD
\begin{columns}
	\column{0.5\linewidth}
	\begin{figure}
		\includegraphics[width=\linewidth]{images/SimKRAJJAk140.pdf}
		\caption{$\xi < 0$}
	\end{figure}
	\column{0.5\linewidth}
	\begin{figure}
		\includegraphics[width=\linewidth]{images/SimHOJJAk150.pdf}
		\caption{$\xi = 0$}
	\end{figure}
\end{columns}
\end{frame}

\begin{frame}
\frametitle{Case study, South}
\begin{columns}
	\column{0.5\linewidth}
	\begin{figure}
		\includegraphics[width=\linewidth]{images/ShapeSouth.pdf}
		\caption{Shape plot}
	\end{figure}
	\column{0.5\linewidth}
	\begin{figure}
		\includegraphics[width=\linewidth]{images/MESouth.pdf}
		\caption{Mean excess plot}
	\end{figure}	
\end{columns}
\end{frame}

\begin{frame}
\frametitle{Case study, South}
\begin{figure}
	\includegraphics[width=0.6\linewidth]{images/trTestSouth.pdf}
	\caption{Truncation test}
\end{figure}
\begin{table}[H]
	\centering
	\tiny
	\begin{tabular}{|c c c c c c c c c c|}
		\hline
		243.9 & 205.7 & 160.3 & 157.2 & 129.4 & 123.3 & 122.0 & 119.7 & 113.7 & 110.8 \\
		\hline
	\end{tabular}
	\caption{Top ten values}
	\label{Top10South}
	\end{table}
\end{frame}

\begin{frame}
\frametitle{Case study, North}
\begin{columns}
	\column{0.5\linewidth}
	\begin{figure}
		\includegraphics[width=\linewidth]{images/ShapeNorth.pdf}
		\caption{Shape plot}
	\end{figure}
	\column{0.5\linewidth}
	\begin{figure}
		\includegraphics[width=\linewidth]{images/MENorth.pdf}
		\caption{Mean excess plot}
	\end{figure}	
\end{columns}
\end{frame}

\begin{frame}
\frametitle{Case study, North}
\begin{figure}
\includegraphics[width=0.6\linewidth]{images/trTestNorth.pdf}
\caption{Truncation test}
\end{figure}
\begin{table}[H]
\centering
\tiny
\begin{tabular}{|c c c c c c c c c c|}
	\hline
	152.2 & 150.8 & 150.6 & 149.0 & 146.8 & 132.9 & 132.4 & 130.1 & 125.1 & 122.0\\
	\hline
\end{tabular}
\caption{Top ten values}
\label{Top10South}
\end{table}
\end{frame}

\section{Conclusions}

\begin{frame}
\frametitle{Conclusions}
\begin{itemize}
	\item A sample size of at least 3000 is needed for accurate detection
	\item The endpoint estimator has a big impact on the performance
	\item Independent of the sign of the shape parameter around 0
	\item There seems to exist a maximum value on convective rainfall in the north of Ghana
\end{itemize}
\end{frame}


\end{document}